\documentclass[a4paper,12pt]{article}
\usepackage[utf8]{inputenc}
\usepackage[spanish]{babel}
\usepackage{amsmath}
\usepackage{graphicx}
\usepackage{hyperref}
\usepackage{float}

\title{Análisis de la Distribución Horaria de Incidentes}
\author{Tu Nombre}
\date{\today}

\begin{document}

\maketitle

\section{Introducción}

Se estudió la distribución de incidentes delictivos a lo largo del día, agrupando los datos por hora (0 a 23). El objetivo fue modelar esta variación temporal para entender patrones periódicos que pudieran relacionarse con horarios de mayor riesgo.

\section{Visualización Inicial}

Se obtuvo un histograma de conteos por hora \( y_i \) para cada hora \( x_i \in \{0,1,\ldots,23\} \), mostrando una clara periodicidad diaria:

\[
(x_i, y_i), \quad i=0, \ldots, 23
\]

\section{Modelado con Funciones Senoidales}

Debido a la naturaleza periódica, se decidió aproximar la serie de datos con funciones seno, que capturan patrones repetitivos.

\subsection{Modelo Senoidal Simple}

La forma básica del modelo es:

\[
y(x) = a \sin(bx + c) + d
\]

donde:

\begin{itemize}
    \item \( a \) es la amplitud,
    \item \( b \) la frecuencia angular,
    \item \( c \) el desfase (fase),
    \item \( d \) un desplazamiento vertical (offset).
\end{itemize}

Se ajustó este modelo con parámetros iniciales basados en la periodicidad diaria esperada \( \left(b \approx \frac{2\pi}{24}\right) \).

\subsection{Modelo con Dos Términos Senoidales}

Para capturar patrones más complejos (por ejemplo, dos picos en el día), se extendió el modelo a:

\[
y(x) = a_1 \sin(b_1 x + c_1) + a_2 \sin(b_2 x + c_2) + d
\]

donde:

\[
b_1 = \frac{2\pi}{24} \quad \text{(frecuencia diaria)}, \quad b_2 = \frac{4\pi}{24} \quad \text{(doble frecuencia diaria)}.
\]

Este modelo mejoró significativamente la explicación de la variabilidad en los datos, con un coeficiente de determinación \( R^2 \approx 0.82 \).

\subsection{Modelo con Tres Términos Senoidales}

Finalmente, se propuso un modelo aún más flexible con tres términos:

\[
y(x) = a_1 \sin(b_1 x + c_1) + a_2 \sin(b_2 x + c_2) + a_3 \sin(b_3 x + c_3) + d
\]

con frecuencias:

\[
b_1 = \frac{2\pi}{24}, \quad b_2 = \frac{4\pi}{24}, \quad b_3 = \frac{6\pi}{24}.
\]

Este modelo alcanzó un ajuste excelente con:

\[
R^2 = 0.94, \quad \text{RMSE} = 85.34,
\]

lo que indica que el modelo explica el 94\% de la variabilidad de los incidentes por hora.

\section{Evaluación del Modelo}

Para medir qué tan bien el modelo ajusta los datos se usaron:

\begin{itemize}
    \item \textbf{Coeficiente de determinación \( R^2 \):}

    \[
    R^2 = 1 - \frac{\sum_{i=0}^{23} (y_i - \hat{y}_i)^2}{\sum_{i=0}^{23} (y_i - \bar{y})^2}
    \]

    donde \( \hat{y}_i \) es el valor predicho y \( \bar{y} \) la media de los datos.

    \item \textbf{Raíz del error cuadrático medio (RMSE):}

    \[
    RMSE = \sqrt{\frac{1}{24} \sum_{i=0}^{23} (y_i - \hat{y}_i)^2}
    \]
\end{itemize}

Ambas métricas indican un buen ajuste, con valores cercanos a 1 para \( R^2 \) y un RMSE razonable dado el rango de datos.

\section{Visualización de Resultados}

A continuación se muestran los gráficos con los datos reales y la curva ajustada.

\begin{figure}[H]
    \centering
    % Inserta aquí tu gráfico exportado (png, jpg, pdf...)
    \includegraphics[width=0.9\textwidth]{ruta/a/tu/grafico.png}
    \caption{Distribución de incidentes por hora (barras) y modelo ajustado (línea).}
\end{figure}

\section{Conclusión}

El análisis demuestra que la frecuencia de incidentes delictivos a lo largo del día sigue un patrón periódico que puede modelarse eficazmente con una suma de funciones senoidales. Este modelo permite identificar las horas de mayor riesgo (picos) y puede apoyar decisiones en prevención y vigilancia.

\vspace{1em}
\textbf{Opcional:} Se pueden explorar modelos aún más complejos o incluir variables adicionales para mejorar el ajuste.

\end{document}

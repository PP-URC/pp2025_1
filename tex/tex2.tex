\documentclass[a4paper,12pt]{article}
\usepackage[utf8]{inputenc}
\usepackage[spanish]{babel}
\usepackage{amsmath}
\usepackage{graphicx}
\usepackage{hyperref}
\usepackage{float}
\usepackage{booktabs}

\title{Análisis de la Distribución Horaria de Incidentes}
\author{}
\date{\today}

\begin{document}

\maketitle

\section{Introducción}

Se analizó la frecuencia de incidentes delictivos registrados en el Metro de la Ciudad de México, agrupándolos por hora del día (0 a 23). El objetivo fue modelar su variación horaria para entender patrones periódicos y proponer interpretaciones útiles para prevención o intervención.

\section{Visualización Inicial}

A partir de los datos, se generó un histograma de conteos por hora \( y_i \) para cada hora \( x_i \in \{0,1,\ldots,23\} \). Este histograma sugiere una distribución no uniforme, con algunos picos a determinadas horas:

\begin{figure}[H]
    \centering
    \includegraphics[width=0.9\textwidth]{images/histograma_incidentes.png}
    \caption{Histograma de incidentes por hora.}
\end{figure}

\section{Modelado con Polinomios}

Como primer enfoque se ajustó un modelo polinomial de grado 4 utilizando mínimos cuadrados:

\[
\hat{y}(x) = c_0 + c_1 x + c_2 x^2 + c_3 x^3 + c_4 x^4
\]

Este tipo de modelo no asume periodicidad, pero puede capturar curvaturas generales en la serie. Se utilizó la función \texttt{Polynomial.fit} de NumPy.

\begin{itemize}
    \item \( R^2 = 0.5058 \)
    \item RMSE = 123.65
\end{itemize}

\begin{figure}[H]
    \centering
    \includegraphics[width=0.9\textwidth]{images/polynomial_fit.png}
    \caption{Ajuste con modelo polinomial de grado 4.}
\end{figure}

Aunque el modelo logra cierta aproximación, su capacidad predictiva es limitada: explica solo el 50\% de la variabilidad y presenta errores más altos que otros enfoques.

\section{Modelado con Funciones Senoidales}

Dado que los incidentes tienden a seguir patrones diarios (por ejemplo, más actividad en horarios laborales), se empleó un modelo basado en la suma de funciones seno.

\subsection{Modelo senoidal de tres términos}

El modelo final incorpora tres términos senoidales con diferentes frecuencias para capturar tanto el ciclo diario como posibles repeticiones intra-día:

\[
\hat{y}(x) = a_1 \sin(b_1 x + c_1) + a_2 \sin(b_2 x + c_2) + a_3 \sin(b_3 x + c_3) + d
\]

Con:

\[
b_1 = \frac{2\pi}{24}, \quad b_2 = \frac{4\pi}{24}, \quad b_3 = \frac{6\pi}{24}
\]

\begin{itemize}
    \item \( R^2 = 0.9411 \)
    \item RMSE = 42.67
\end{itemize}

\begin{figure}[H]
    \centering
    \includegraphics[width=0.9\textwidth]{images/sinoidal.png}
    \caption{Ajuste del modelo senoidal de tres términos.}
\end{figure}

\begin{figure}[H]
    \centering
    \includegraphics[width=0.9\textwidth]{images/residuals.png}
    \caption{Residuos del modelo senoidal.}
\end{figure}

Este modelo captura adecuadamente los picos y valles del comportamiento diario de incidentes.

\section{Comparación de Modelos}

Se resumen a continuación las métricas de evaluación para ambos modelos:

\begin{table}[H]
    \centering
    \begin{tabular}{lcc}
        \toprule
        \textbf{Modelo} & \textbf{\( R^2 \)} & \textbf{RMSE} \\
        \midrule
        Polinomial (grado 4) & 0.5058 & 123.65 \\
        Senoidal (3 términos) & 0.9411 & 42.67 \\
        \bottomrule
    \end{tabular}
    \caption{Comparación entre modelos}
\end{table}
